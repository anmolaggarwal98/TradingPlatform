\subsubsection{Monte Carlo Strategy}\label{monte-carlo-strategy}

Brownian Motion

The process $\{W(t)\}_{t\geq0}$ is said to be (standard) Brownian Motion
if the follwoing are satisfied:

$W(0)=0$

For $s,t\geq0$ the random variable $W(s+t)-W(s) \sim N(0,t)$

Whenever $0\leq t_0\leq t_1\leq....<t_n$ , the quantities
$W(t_1)-W(t_0),W(t_2)-W(t_1),....,W(t_n)-W(t_{n-1})$ are
\emph{independent}

$W(t)$ is a \emph{continuous} function of $t$ with probability $1$

\subparagraph{Deriving Ito's Lemma (\emph{Time
Independent})}\label{deriving-itos-lemma-time-independent}

Let us suppose that the asset price $S$ satisifies the
\textbf{Stochastic Differential Equation (SDE)}

\[ dS = \mu dt + \sigma dW \]

where $\mu (t)$ and $\sigma(t)$ depends on the time-interval we look at
(say 10D period) and the $W(s)$ for $s\leq t$ is a Brownian motion i.e
the random fluctuation in a stock-price $S$

Now consider a function $f(S(t),t)$ of asset price where $f$ has a
\emph{continuoous second derivative} (i.e $f\in C^2 (0,T)$). For
simplicity, let us assume that $f$ is \emph{independent} of time i.e
$f=f(S)$. Then by \textbf{Taylor's Theorem}:

\[ f(S+dS) = f(S) + f'(S)dS + \frac{1}{2}f''(S)(dS)^2+o((dS)^2) \]

Now:
$dS = \mu dt + \sigma dW \\ (dS)^2 = \mu^2(dt)^2+2\mu\sigma dt*dW + \sigma^2(dW)^2$

But since $dW$ has order $\sqrt{dt}$ then it has order $dt$. So overall
we have: \[ (dS)^2 = \sigma^2(dW)^2 + o(dt)\]

After suitable subsitution, we overall get that:

\[ f(S+dS) - f(S) = f'(S)[\mu dt+\sigma dW] + \frac{1}{2}f''(S)\sigma^2(dW)^2 + o(dt) \]
\textless{}\p\textgreater{}

Since $W(t)$ is a Brownian Motion $\implies$ $dW=W(t+dt) - W(t)$ is a
Brownian Motion with
$dW \sim N(0,dt) \\ \implies E[(dW)^2]= Var(dW)+0 \\ \implies =dt$

So in the limit and replacing $(dW)^2$ by $dt$ , we get:

\[df = \frac{df}{dS}(\mu dt+\sigma dW) + \frac{1}{2}\frac{d^2f}{dS^2}\sigma^2dt\]

Ito's Lemma (Time Independent)

Let $f(S)$ be a continuous twice differentiable and suppose that:
\[ dS = \mu dt + \sigma dW\] Then:
\[df = \frac{df}{dS}(\mu dt+\sigma dW) + \frac{1}{2}\frac{d^2f}{dS^2}\sigma^2dt\]
When written out in the Integral form:
\[ f(S(T)) - f(S(0)) = \int_{0}^{T} \left(\mu dt+\sigma dW\right) dt + \int_{0}^{T} \sigma\frac{df}{dS} dW\]

Thus we get a relationship between a \emph{Stochastic Integral} and a
\emph{Standard Integral} with respect to \emph{time}.

\subparagraph{A model for stock price}\label{a-model-for-stock-price}

Consider an asset with price $S(t)$ that evolves according to the SDE
\[ dS = \mu Sdt + \sigma SdW\]

Over a period $dt$, the price changes by a \textbf{deterministic
quantity} $\mu Sdt$ (representing some underlying deterministic growth)
and a \textbf{random quantity} $\sigma SdW$ (where \$\sigma\$ measures
the volatility of the asset).

It is useful to work in terms of $log(S(t))$ so we define:
\[ f(S) = log(S) \\ f'(S) = \frac{1}{S} \\ f''(S) = -\frac{1}{S^2} \]

So after plugging the following in the SDE we get:

\[ df = \left(\mu - \frac{1}{2}\sigma^2\right) + \sigma dW \]

Plugging into \textbf{Ito's Lemma} we get:

\[ log(S(T)) - log(S(0)) = \left(\mu - \frac{1}{2}\sigma^2\right)T + \sigma W(T) \quad (*)\]

We conclude that:

\[ log\left(\frac{S(T)}{S(0)}\right) \sim N\left((\mu - \frac{1}{2}\sigma^2)T,\sigma^2T\right) \]
The above equation can be generalised to give the following:

\[ log\left(\frac{S(t+\Delta t)}{S(t)}\right) \sim N\left((\mu - \frac{1}{2}\sigma^2)\Delta t,\sigma^2\Delta t\right)\]

So we say that $Y = \frac{S(t)}{S(0)}$ is \textbf{log-normally
distributed} with:
\[ E[Y] = exp\left[{\eta+\frac{1}{2}\sigma^2}t\right] \\ \] where
$\eta = (\mu - \frac{1}{2}\sigma^2)t$

Using $(*)$ we can show that:

\[ S(t+\Delta t) = S(t)*exp\left[\left(\mu - \frac{1}{2}\sigma^2\right)\Delta t + \sigma W(t)\right]\]

where $\mu$ is \textbf{1+log-return's mean} and $\sigma$ is
\textbf{1+log-return's standard deviation}

In my algorithm:
\[ \Delta t = 1 \quad \text{since I have daily data} \\ 
    S_{t+1} = S_t*exp\left[\left(\mu - \frac{1}{2}\sigma^2\right) + \sigma W_t\right] \]

\begin{verbatim}
$$ffd$$




























































g
\end{verbatim}
